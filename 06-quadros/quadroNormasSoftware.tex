\begin{quadro}[htb]
    \begin{center}
        
   
    \caption[Normas para qualidade do software]{Normas para qualidade do software.\label{quad:NormasSoftware}}
\begin{tabular}{|cp{11cm}|}
	\hline
	\textbf{Norma} & \textbf{Comentário} \\ \hline
	ISO 9126 & Características da qualidade de produtos de software.   \\ \hline
	
	NBR 13596 & Versão brasileira da ISO 9126    \\ \hline
	
	ISO 14598 & Guias para a 
            avaliação de produtos de software, 
            baseados na utilização prática da norma ISO 9126    \\ \hline
	
	ISO 12119 & Características de qualidade de pacotes de software (software de
prateleira, vendido com um produto embalado)    \\ \hline
	
	IEEE P1061 & Standard for Software Quality Metrics Methodology (produto de
software)   \\ \hline
	
	ISO 12207 & Software Life Cycle Process. Norma para a qualidade do processo de desenvolvimento de software.    \\ \hline
	
	NBR ISO 9001  & Sistemas de qualidade - Modelo para garantia de qualidade em Projeto,
Desenvolvimento, Instalação e Assistência Técnica (processo)    \\ \hline
	
	NBR ISO 9000-3 & Gestão de qualidade e garantia de qualidade. Aplicação da norma ISO
9000 para o processo de desenvolvimento de software    \\ \hline
	
	NBR ISO 10011 & Auditoria de Sistemas de Qualidade (processo)    \\ \hline
	
	CMM & Capability Maturity Model. Modelo da SEI para avaliação da
qualidade do processo de desenvolvimento de software. Não é uma
norma ISO, mas é muito bem aceita no mercado.    \\ \hline
	
	SPICE & Projeto da ISO/IEC para avaliação de processo de desenvolvimento de
software. Ainda não é uma norma oficial ISO, mas o processo está em
andamento.   \\ \hline
	
	 
\end{tabular}
	\end{center}
	\vspace*{-0,8cm}

	{\raggedright \fonte{\cite{UNEMAT}}}
	
\end{quadro}

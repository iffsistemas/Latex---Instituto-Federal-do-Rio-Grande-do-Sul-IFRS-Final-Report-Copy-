%
% Documento: Disposições
%

\chapter{REFERENCIAL TEÓRICO}\label{chap:referencial}

Neste capítulo iremos entender os conceitos de qualidade e qualidade de \textit{software}; veremos os ciclos de desenvolvimento de um \textit{software} e a importância de escolher um ciclo adequado; e, como funciona a técnica - que norteará o desenvolvimento do \textit{software} proposto neste trabalho - de garantia e qualidade qualidade RTF - Revisão Técnica Formal.

\section{Qualidade}


Qualidade, na definição do dicionário Michaelis \footnote{https://michaelis.uol.com.br/moderno-portugues/busca/portugues-brasileiro/qualidade/}, significa "atributo, condição natural ou propriedade pela qual algo ou alguém se individualiza, distinguindo-se dos demais". Derivada do latin \textit{qualitate}, a palavra também pode expressar "precisão, o grau de perfeição ou conformidade a certo padrão". Em suma, como corrobora o Dicionário Aurélio\footnote{https://dicionariodoaurelio.com/qualidade}, "é a maneira de ser de algo ou alguém, sua superioridade ou excelência".

No decorrer da história, vários autores tentaram definir o real significado de qualidade. Para \citeonline{GARVIN2002}, esse termo dispõe de várias interpretações e por isso "é essencial um melhor entendimento do termo para que a qualidade possa assumir um papel estratégico". Muitos procuraram defini-lá de acordo com seus diversos pontos de vista. Na visão generalista de Philip Crosby, qualidade é apenas atendimento aos requisitos. Ele defende a ideia de que não se deve preocupar com percepções subjetivas de qualidade, e sim focar-se no atendimento aos requisitos e especificações do produto \cite{CROSBY1979}.

Para \citeonline{CAMPOS1992}, qualidade é a adequação ao uso, é a totalidade das características de um produto ou serviço que se relacionam com sua capacidade de atender às necessidades de um cliente. Qualidade é o grau no qual um conjunto de características inerentes satisfaz aos requisitos. \cite{ISO90002005}

\begin{citacao}

A maioria das pessoas concorda que qualidade é aquilo que produz satisfação, que está relacionada a um preço justo, a um produto que funciona corretamente e a um serviço prestado de forma a superar as expectativas de quem dela faz uso. \cite[p. 52]{VERGUEIRO2002} 

\end{citacao}

Na relação entre qualidade, produto e cliente final, \citeonline{DEMING1986} sustenta que qualidade de um produto é definida por meio da percepção do cliente final daquele produto, portanto, por mais que um produto possa atender a todos os requisitos técnicos e ser vendido por um preço justo, se não for valorizado pelo cliente, não tem qualidade.

A defesa de \citeonline{ARAUJO2007} é que a qualidade é busca pela perfeição, almejando agradar aos clientes que estão cada vez mais exigentes. "Qualidade é adequação ao uso, é aquilo que não cria problemas, é fazer a coisa certa na primeira vez." \cite[p. 15]{RANGEL1195}

Assim como existem diversos conceitos, a qualidade possui também inúmeros objetivo. Esses objetivos estendem-se ao produto: (variáveis e atributos que podem ser medidos e controlados); ao usuário: (produto é o que o cliente compra); à fabricação: (adequação às normas e às especificações); ao valor: (adequado ao uso e ao preço). \cite{MARTINSELAUGENI2006}



%objetivo da qualidade

%padrões e normas para garantir a qualidade

\section{Qualidade de Software}

\citeonline{PRESSMAN2002} define qualidade de \textit{software} como a conformidade com os requisitos; com os padrões de desenvolvimento e com as características implícitas que são esperadas em todo \textit{software} desenvolvido profissionalmente. 
 
 A definição de Alexandre Bartié vai de encontro ao de Pressaman: "[...] qualidade de \textit{software} é um processo sistemático que focaliza todas as etapas e artefatos produzidos com o objetivo de garantir a conformidade de processos e produtos, prevenindo e eliminando defeitos". \cite[p. 12]{BARTIE2002}
 
 
 Diante dessas definições, podemos sintetizar que a qualidade de \textit{software} é um conjunto de exigências e/ou condições que o \textit{software} deverá possuir para que o produto atenda às necessidades e satisfaça as partes interessadas, sejam elas: empresas, organizações ou clientes.
 
 Para conseguir lucrar e vender mais, a qualidade não é apenas um diferencial de mercado, ela tem se tornado um pré-requisito na qual a empresa deve conquistar para colocar seu produto no Mercado Global. A existência da qualidade não é um fator de vantagem no mercado, mas sim uma necessidade para a garantia da competitividade. \cite{PRESSMAN2004}
 
 A conquista pela qualidade não é fácil. O desenvolvimento de \textit{software} com elevada produtividade, dentro do prazo estabelecido, sem necessitar de mais recursos do que aqueles alocados, assegurando um \textit{software} de qualidade, tem sido um desafio \cite{SOMMERVILLE2003}.


À medida que o \textit{software} passou a se tornar cada vez mais integrado à vida cotidiana, a preocupação com a qualidade cresceu. Um bom processo de desenvolvimento de \textit{software} deve capacitar a organização à construção de produtos de boa qualidade. À vista disso, existem dois tipos de qualidades a serem padronizadas, uma referente ao produto (\textit{software}) e o outro referente ao processo. \cite{PRESSMAN2011}

Apenas para ter uma uma visão geral, observe o quadro
abaixo com as principais normais nacionais e internacionais para permitir a correta avaliação de qualidade tanto de produtos de \textit{software} quanto de processos de
desenvolvimento de \textit{software}:

\begin{quadro}[htb]
    \begin{center}
        
   
    \caption[Normas para qualidade do software]{Normas para qualidade do software.\label{quad:NormasSoftware}}
\begin{tabular}{|cp{11cm}|}
	\hline
	\textbf{Norma} & \textbf{Comentário} \\ \hline
	ISO 9126 & Características da qualidade de produtos de software.   \\ \hline
	
	NBR 13596 & Versão brasileira da ISO 9126    \\ \hline
	
	ISO 14598 & Guias para a 
            avaliação de produtos de software, 
            baseados na utilização prática da norma ISO 9126    \\ \hline
	
	ISO 12119 & Características de qualidade de pacotes de software (software de
prateleira, vendido com um produto embalado)    \\ \hline
	
	IEEE P1061 & Standard for Software Quality Metrics Methodology (produto de
software)   \\ \hline
	
	ISO 12207 & Software Life Cycle Process. Norma para a qualidade do processo de desenvolvimento de software.    \\ \hline
	
	NBR ISO 9001  & Sistemas de qualidade - Modelo para garantia de qualidade em Projeto,
Desenvolvimento, Instalação e Assistência Técnica (processo)    \\ \hline
	
	NBR ISO 9000-3 & Gestão de qualidade e garantia de qualidade. Aplicação da norma ISO
9000 para o processo de desenvolvimento de software    \\ \hline
	
	NBR ISO 10011 & Auditoria de Sistemas de Qualidade (processo)    \\ \hline
	
	CMM & Capability Maturity Model. Modelo da SEI para avaliação da
qualidade do processo de desenvolvimento de software. Não é uma
norma ISO, mas é muito bem aceita no mercado.    \\ \hline
	
	SPICE & Projeto da ISO/IEC para avaliação de processo de desenvolvimento de
software. Ainda não é uma norma oficial ISO, mas o processo está em
andamento.   \\ \hline
	
	 
\end{tabular}
	\end{center}
	\vspace*{-0,8cm}

	{\raggedright \fonte{\cite{UNEMAT}}}
	
\end{quadro}


O objetivo, ao desenvolver um produto de \textit{software} e quando esse produto for entregue e realmente usado pelos usuários, não é alcançar a qualidade perfeita, mas a qualidade necessária e suficiente para o uso especificado \cite{MALDONADO2001}. É essencial identificar as características de qualidade necessárias para um determinado
produto de \textit{software} e definir em que grau essas características precisam ser alcançadas para satisfazer às necessidades dos usuários.

Caracteriza-se, segundo \citeonline{MALDONADO2001}, um produto de \textit{software},  :

\begin{itemize}
    \item Funcionalidade: refere-se à existência de um conjunto de funções, que satisfazem às
necessidades explícitas ou implícitas e suas propriedades específicas;

    \item Confiabilidade: refere-se à capacidade de o \textit{software} manter seu nível de
desempenho sob condições estabelecidas, por um período de tempo;
    \item  Usabilidade: refere-se ao esforço necessário para usar um produto de \textit{software}, bem
como o julgamento individual de tal uso por um conjunto explícito ou implícito de
usuários;
    \item  Eficiência: refere-se ao relacionamento entre o nível de desempenho do \textit{software} e a
quantidade dos recursos utilizados sob as condições estabelecidas;
    \item  Manutenibilidade: refere-se ao esforço necessário para fazer modificações
específicas no \textit{software};
    \item  Portabilidade: refere-se à capacidade do \textit{software} ser transferido de um ambiente
para o outro;
    \item  Efetividade: refere-se à capacidade do produto de \textit{software} possibilitar aos usuários
atingir metas especificadas com acurácia e completeza em um contexto de uso
especificado;
    \item  Produtividade: refere-se à capacidade de possibilitar aos usuários utilizar uma
quantidade adequada de recursos;
    \item  Segurança: refere-se à capacidade de oferecer níveis aceitáveis de risco de danos a
pessoas, negócios, \textit{software}, propriedade ou ao ambiente especificado;
    \item Satisfação: refere-se à capacidade de satisfazer os usuários em um contexto de uso especificado.
\end{itemize}

\section{Software}

Mas afinal, o que é \textit{software}? \textit{Software} é um conjunto composto por instruções de computador, estruturas de dados e documentos \cite{PRESSMAN2006}; é como um elemento de sistema lógico, e não físico que não se desgasta \cite{PRESSMAN2002}; é caracterizado como um programa de computador e toda a documentação associada a ele \cite{SOMMERVILLE2003}.

\citeonline{PRESSMAN1995} ainda detalha \textit{software} como sendo: 

\begin{enumerate}
    \item Instruções que quando executadas produzem a função e o desempenho desejados;
    \item Estruturas de dados que possibilitam que os programas manipulem adequadamente a informação;
    \item Documentos que descrevem a operação e uso dos programas.
\end{enumerate}
    

\begin{citacao}
Ao passar do tempo, ninguém imaginava que o \textit{software} tornaria um elemento muito importante para o mundo e teria a capacidade de manipular a informação. Com muitos elementos computacionais tiveram mudanças até hoje e continuam tendo. Com este crescimento computacional, levam a criação de sistemas perfeitos e problemas para quem desenvolve \textit{software}s
complexos. As preocupações dos engenheiros de \textit{software} para
desenvolverem os \textit{software}s sem defeitos e entregarem estes produtos no tempo marcado, assim leva a aplicação da disciplina de engenharia de \textit{software}. Com o crescimento desse segmento muitas empresas possuem mais especialistas em TI em que cada um tem sua responsabilidade no desenvolvimento de \textit{software} e é diferente de antigamente que era um único
profissional de \textit{software} que trabalhava sozinho numa sala \cite[p. 39]{PRESSMAN2007}.
\end{citacao}

Para \cite[p. 31]{PRESSMAN2011}, o \textit{software}: 1) "transforma dados pessoais (por exemplo, transações financeiras de um indivíduo) de modo que possam ser mais úteis num determinado contexto; 2) "gerencia informações comerciais para aumentar a competitividade"; 3) "fornece um portal para redes mundiais de informação (Internet) e os meios para obter informações sob todas as suas formas. "O \textit{software} distribui o produto mais importante de nossa era – a
informação".

\subsection{Tipos de Software}

Software Básico - são aqueles que dão base a outros programas. Algumas características são:  a intensa interação com o hardware e compartilhamento de recursos; uso constante de processamento concorrente, que demanda o escalonamento, e estruturas de dados muito complexas. São exemplos: compiladores, editores de texto, sistemas operacionais \cite{PRESSMAN2007}.

Software de Tempo Real - monitora, avalia e controla fatos do mundo real. As características desse tipo é: coleta de dados do ambiente externo; transformar a informação de acordo com a necessidade do sistema; controlar a saída para o ambiente externo e um componente de monitoração que coordena todos os outros. Exemplos: aeronaves os controles de navegação, nos automóveis os sistemas de injeção eletrônica. \cite{PRESSMAN2007}.

Software Comercial - segundo \citeonline{PRESSMAN2007}, essa é a maior área privada de \textit{software}. Nessa categoria os dados são reunidos de uma forma que facilite as operações comerciais e as decisões administrativas. Temos exemplos os controle de estoque, folha de pagamento, contas a pagar e receber.

Software de Computador Pessoal - são os
responsáveis por processamento de textos, planilhas eletrônicas, computação gráfica \cite{MODESTOEOLIVEIRA2010}.

Software Científico e de Engenharia - De acordo com \citeonline{MODESTOEOLIVEIRA2010}, são \textit{software} que auxiliam as aplicações científicas. Têm sido caracterizados por algoritmos de processamento de números”.

Software Embutido ou Embargado - como o próprio nome sugere, são \textit{software} próprios de um determinado hardware. É usado para controlar produtos e sistemas para os mercados industriais e de consumo. Tem como característica utilizar uma memória de somente leitura e usam rotinas limitadas e particulares \cite{MODESTOEOLIVEIRA2010}.

Software de Inteligência Artificial - Para \citeonline{MODESTOEOLIVEIRA2010} caracteriza-se pelo uso de algoritmos não numéricos para resolver problemas complexos. Atualmente a área de AI (\textit{Artificial Intelligency}) mais ativa é a dos sistemas especialistas, também chamados sistemas baseados em conhecimento.

Software Online - são \textit{software} que trabalham em conexão com a internet. Os arquivos não são carregados localmente e sim através de um servidor, com tempo de resposta curto, mas maior que o de tempo real \cite{MODESTOEOLIVEIRA2010}.


\subsection{Ciclo de vida do Software}

O ciclo de vida do \textit{software} consiste de todas as atividades desde a criação até a sua “aposentadoria”. Esse ciclo foi pensado e criado para tornar a atividade de desenvolvimento do \textit{software} menos problemática e visa organizar o desenvolvimento utilizando técnicas e métodos \cite{SOMMERVILLE2008}.

Decidir qual o modelo de ciclo de vida utilizar é uma das mais importantes e influentes decisões para o sucesso do projeto. Do contrário, escolher um modelo inadequado ou a falta de um pode fazer que haja muito retrabalho e frustração ao longo do ciclo de vida do \textit{software} \cite{MCCONNELL1996}.

São descritos, a seguir,  os modelos mais utilizados \cite{SOMMERVILLE2008}, \cite{PRESSMAN2006}.

\subsubsection{Modelo Em Cascata}
\subsubsection{Modelo Incremental}
\subsubsection{Modelos Evolucionários}
\subsubsection{Modelo De Prototipagem}
\subsubsection{Modelo Espiral}
















%qualidade do produto e a qualidade do processo.
%0 QUE É SOFTWARE
%Confiabilidade de Software
%Garantia de Qualidade
%Processos e a Qualidade
%Garantia da Qualidade do Software
%Organização Internacional de Padronização (ISO)
%Processos do Ciclo de Vida do Software
%Automatização de Testes
%RTF



\section{MPS.BR}






\section{Revisão técnica formal (RTF)}